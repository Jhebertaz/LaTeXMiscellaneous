%%%%%%%%%%%%%%%%%%%%%%%%%%%%%%%%%%%%%%%%%%%%%%%%%%%%%%
%% Auteur	: Julien Hébert-Doutreloux
%% Date		: 2020/05/01
%% Mail		: 
%% Git		: https://github.com/Jhebertaz/
%%%%%%%%%%%%%%%%%%%%%%%%%%%%%%%%%%%%%%%%%%%%%%%%%%%%%
%% Réference
%% https://en.wikibooks.org/wiki/LaTeX
%% https://fr.wikibooks.org/wiki/LaTeX
%% https://www.overleaf.com/
%%%%%%%%%%%%%%%%%%%%%%%%%%%%%%%%%%%%%%%%%%%%%%%%%%%%%
\documentclass[french, babel]{article}
%% Pour le français
\usepackage [french]{babel}
\usepackage [utf8]{inputenc}
\usepackage [T1] {fontenc}
\usepackage{xspace}
%% Marge
\usepackage{layout}
%% Mise en page
\usepackage[a4paper]{geometry}
%% American Mathematical Society Package 
\usepackage{amsmath}
\usepackage{amsfonts}
\usepackage{amssymb}
\usepackage{amsthm}
%% Divers 
\usepackage{graphicx}
\usepackage{caption}
\usepackage{wrapfig}
\usepackage{framed}
\usepackage{lipsum}
\usepackage{multicol}
\usepackage{listings}
\usepackage{hyperref}
%% Information
\title{Un gabaraits \LaTeX}
\author{Julien H\'ebert-Doutreloux\\%
	Commité de l'Axiomatique%
	\thanks{D\'epartement de Math\'ematique}, Université de Montréal}
\date{\today}
\begin{document}
	%% Affiche la page présentation
	\maketitle
	%% Table des matière
	\tableofcontents
	%% Espace vertical
	\vspace{1cm}
\section[Ce qui apparait dans la table des mati\`ere]{Une section en une colonne}
	%% Du blabla
	\lipsum[3-5]
	%% Commande pour passer en deux colonnes
\section[D\'ebut d'article]{Une section en deux colonnes}
	\begin{multicols*}{2}
		\lipsum[5-6]
			\[\frac{te^{tx}}{e^t-1}=B_0(x)+B_1(x)t+B_3(x)\frac{t^3}{3!}+\dots\]
			\captionof{figure}{Le polyn\^ome de Bernoulli}
		\lipsum[7-8]
		%% Utilisation spécial pour l'insertion d'image dans un environnement en deux colonne
		\begin{center}
			\includegraphics[width=.3\textwidth]{WordCloudLaTeX}
			\caption{Un nuage de mot en lien avec \LaTeXe}
		\end{center}
	\lipsum[8-14]
	\end{multicols*}
\section[Suite avec des tableaux]{Une complexit\'ee matricielle}
	\lipsum[14-15]
	%% La configuration ht pour placer la figure à la sa place (se qui est avant reste avant)
	\begin{figure}[ht]
	\centering
	$
	\left(
	\begin{array}{cccc}
		1.93409 + i & -2.05004+ i &
		8.23347 + i & 2.98992 + i \\
		-1.59947+ i & -5.25552+ i &
		1.44115 + i & -6.03481+ i \\
		-6.95121+ i & -8.37939+ i &
		-4.32323+ i & 7.91667 + i \\
		-5.70742+ i & -0.322354+i &
		-3.23434+ i & 3.29176 + i \\
	\end{array}
	\right)
	$
	\caption{Un sous-titre pour une matrice aléatoire complexe}
	\end{figure}
\subsection[Sous-section]{Comment coder}
	\lipsum[15-16]
	%% Pour inserer un code
	\begin{lstlisting}[language=Mathematica, caption=Exemple Mathematica]
	(*Pour importer le texte de la page web*)
	text = Import["https://en.wikipedia.org/wiki/LaTeX"];
	(*Exporte le nuage de mot sous l'extension .png*)
	Export["WordCloudLaTeX.png", WordCloud[text]]
	\end{lstlisting}
	\begin{multicols}{3}
		\lipsum[16-17]
	\end{multicols}
	\begin{lstlisting}[language=Python,caption={Exemple Python\protect\footnotemark\relax}]
		list_of_lists = [ [1, 2, 3], [4, 5, 6], [7, 8, 9]]
		for list in list_of_lists:
		for x in list:
		print(x)
	\end{lstlisting}
	\footnotetext{\url{https://www.overleaf.com/learn/latex/code_listing}}
	\begin{multicols}{3}
		\lipsum[17-18]
	\end{multicols}
\section[Les d\'echets  FMA-VC]{Les d\'echets faibles ou Moyenne activit\'e \`a Vie Courte}
	On parle de d\'echet FMA-VC pour les radionucl\'eides qui ont des p\'eriodes de demi-vie de moins de trente-et-un ans. Les d\'echets de tr\`es faibles activit\'es (TFA) et les FMA-VC repr\'esentent la grande majorit\'e de l'inventaire de l'ANDRA\footnotetext{\url{https://www.andra.fr/sites/default/files/2020-02/Andra-MAJ_Essentiels_2020-Web.pdf}}.
\pagebreak

{
	\let\footnotemark\relax
	\lstlistoflistings
}
	\listoffigures
	
\end{document}